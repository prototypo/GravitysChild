%%%%%%%%%%%%%%%%%%%%%%%% HAUB.tex %%%%%%%%%%%%%%%%%%%%%%%
%
% Root file for: Gravity's Child (D. Hyland-Wood)
%
%%%%%%%%%%%%%%%%%%%%%%%%%%%%%%%%%%%%%%%%%%%%%%%%%%%%

%\documentclass[10pt]{book} % 10 point font
%\documentclass[11pt]{book} % 11 point font
\documentclass[12pt]{book} % 12 point font
\usepackage[a4paper, top=3cm, bottom=3cm]{geometry}
\usepackage[utf8]{inputenc}
%\usepackage[latin1]{inputenc}
\usepackage{setspace}
\usepackage{fancyhdr}
\usepackage{tocloft}

% Extension Packages
%-------------------------------------------------------------------------------
\usepackage{mathptmx}	% selects Times Roman as basic font
\usepackage{helvet}		% selects Helvetica as sans-serif font
\usepackage{courier}	% selects Courier as typewriter font
%\usepackage{type1cm}	% activate if the above 3 fonts are 
					% not available on your system
					
% From Stack Exchange for ToDo items
% --------------------------------------------------
\usepackage{xargs} 
\usepackage[pdftex,dvipsnames]{xcolor}  % Coloured text etc.
% 
\usepackage[colorinlistoftodos,prependcaption,textsize=tiny]{todonotes}
\newcommandx{\unsure}[2][1=]{\todo[linecolor=red,backgroundcolor=red!25,bordercolor=red,#1]{#2}}
\newcommandx{\change}[2][1=]{\todo[linecolor=blue,backgroundcolor=blue!25,bordercolor=blue,#1]{#2}}
\newcommandx{\info}[2][1=]{\todo[linecolor=OliveGreen,backgroundcolor=OliveGreen!25,bordercolor=OliveGreen,#1]{#2}}
\newcommandx{\improvement}[2][1=]{\todo[linecolor=Plum,backgroundcolor=Plum!25,bordercolor=Plum,#1]{#2}}
\newcommandx{\feedback}[2][1=]{\todo[linecolor=Goldenrod,backgroundcolor=Goldenrod!25,bordercolor=Goldenrod,#1]{#2}}
\newcommandx{\thiswillnotshow}[2][1=]{\todo[disable,#1]{#2}}
					
\usepackage{titling}
\newcommand{\subtitle}[1]{%
  \posttitle{%
    \par\end{center}
    \begin{center}\large#1\end{center}
    \vskip0.5em}%
}

\newenvironment{dedication}
    {\vspace{6ex}\begin{quotation}\begin{center}\begin{em}}
    {\par\end{em}\end{center}\end{quotation}}
    
\usepackage{makeidx}	% allows index generation
\usepackage{graphicx}	% standard LaTeX graphics tool
					% when including figure files
\usepackage{multicol}	% used for the two-column index
\usepackage[bottom]{footmisc}% places footnotes at page bottom

%\usepackage{url}		% Avoid problems with special characters in URLs
\PassOptionsToPackage{hyphens}{url}
\usepackage{hyperref}
\usepackage{breakurl}	% Apparently cannot be used when processing with pdflatex.

\usepackage{soul}		% Allow line breaks in underlined text.

\usepackage{listings}
\lstset{basicstyle=\small, columns=fixed}

\usepackage{amsmath}

% Fix hyphenation
\tolerance=700
\setlength{\emergencystretch}{3em}

% For layout of poetry
\usepackage{lettrine}
\usepackage{parselines}
\usepackage{xcolor}
%\definecolorseries{verso}{rgb}{last}{blue!40!black}{magenta!40!black}
\definecolorseries{verso}{rgb}{last}{black}{black}
\resetcolorseries[30]{verso}
\newenvironment{verso}{\pagebreak[3]\begin{parse lines}[\parindent=1em\noindent]{\color{verso!!+}\hspace{\row\parindent}##1\newline\color{black}}}%
{\end{parse lines}}

%\usepackage{tocbibind}	% Allows the bibliography to appear in the TOC.

\makeindex			% used for the subject index
					% NB: Springer uses the style svind.ist with the makeindex program. 
					% TODO: Should I?

% Use a single bibliography style
%\usepackage[super]{natbib} % Makes citations superscript, which is nicer but conflicts with footnotes. TODO: Change somehow.
\bibliographystyle{plain}

% Custom
%-------------------------------------------------------------------------------
%\usepackage{cyrillic}
\usepackage[T1]{fontenc}   % Allows words with accented characters to be cut-and-pasted, and hyphenated.
\usepackage{enumitem}
\usepackage{phonetic}

% For title page
\newcommand{\HRule}{\rule{\linewidth}{0.5mm}}
\usepackage[export]{adjustbox}

% For Russian (Cyrillic) names in endnotes:
%\newcommand{\cyrrm}{\fontencoding{OT2}\selectfont\textcyrup} % cyrrm = "Roman", or really upright, normal font

\lstset{
      basicstyle = \scriptsize \ttfamily,%
      keywordstyle = [1]\bfseries\color{darkgreen},%
      stringstyle  = \ttfamily\color{darkred},%
      commentstyle = \itshape\color{darkblue},%
      showstringspaces = false,%
%     fancyvrb = true,%
      firstnumber = auto, stepnumber=1, numbersep=5pt,%
      numbers=left, numberstyle=\tiny \ttfamily,
      frame = shadowbox, frameround = ffff, rulesepcolor = \color{shadecolor},
      breaklines=true, breakatwhitespace=true,%
%      prebreak=\textellipsis,postbreak=\textellipsis,%
      emphstyle = \color{red}\underbar, emphstyle = {[2]\color{blue}\underbar},%
      extendedchars = true, inputencoding = utf8,%
%     backgroundcolor=\color{shadecolor},
      xleftmargin=5pt, xrightmargin=2pt,
      captionpos = b
}

\usepackage{eso-pic}
\newcommand\BackgroundPic{%
\put(0,0){%
\parbox[b][\paperheight]{\paperwidth}{%
\vfill
\centering
\includegraphics[width=\paperwidth,height=\paperheight,%
keepaspectratio]{cover/GravitysChild_front.jpg}%
\vfill
}}}
% End Custom
%-------------------------------------------------------------------------------

\begin{document}


% Graphical Cover
%-------------------------------------------------------------------------------
\AddToShipoutPicture*{\BackgroundPic}
\newpage
\thispagestyle{empty}
\mbox{}
\newpage

% Title Pages
%-------------------------------------------------------------------------------
\pagestyle{empty}

\begin{titlepage}

\vspace*{\fill}
\noindent
\textsc{Preview Edition, \today}

\vspace{5 mm}
\noindent
Gravity's Child by David P. Hyland-Wood

\vspace{5 mm}
\noindent
www.hyland-wood.org

\vspace{5 mm}
\noindent
Copyright \textcopyright { }2024 by David P. Hyland-Wood

\vspace{5 mm}
\noindent
Gravity's Child \copy 2024 by David Paul Hyland-Wood is licensed under CC BY-NC-ND 4.0. To view a copy of this license, visit https://creativecommons.org/licenses/by-nc-nd/4.0/

\vspace{5 mm}
\noindent
For additional permission requests, write to the publisher, using ``Attention: Permissions Coordinator,'' at books@hyland-wood.org.

\vspace{5 mm}
\noindent
This is a work of fiction. Names, characters, places, and incidents either are the products of the author's imagination or are used fictitiously. Any resemblance to actual persons, living or dead, businesses, companies, events, or locales is entirely coincidental.

\vspace{5 mm}
\noindent
\textbf{Ebook ISBN}: not yet registered.

\vspace{1 mm}
\noindent
\textbf{Library of Congress Cataloging in Publication Data}: not yet registered.

\end{titlepage}

\pagestyle{empty}

%\pagenumbering{}
% Set book title
%\title{\textbf{Gravity's Child}}
%\subtitle{TODO}
% Include Author name and Copyright holder name
%\author{David Hyland-Wood}
%\maketitle


\begin{titlepage}
\begin{center}

% Upper part of the page. The '~' is needed because \\
% only works if a paragraph has started.
\includegraphics[width=0.80\textwidth, frame]{cover/GravitysChild_front_top}~\\[1cm]

% Title
\HRule \\[0.4cm]
{ \Huge \bfseries Gravity's Child \\[0.4cm] }
\HRule \\[1.5cm]

% Author
\noindent
\begin{minipage}{0.4\textwidth}
\begin{center} \large
\emph{by}\\
David P. Hyland-Wood
\end{center}
\end{minipage}%

\vfill

% Bottom of the page
{\large \today}

\end{center}
\end{titlepage}

% General definitions for all Chapters
%-------------------------------------------------------------------------------
% Define Page style for all chapters
\pagestyle{fancy}
% Delete the current section for header and footer
\fancyhf{}
% Set custom header
\lhead[]{\thepage}
\rhead[\thepage]{}
% Set arabic (1,2,3...) page numbering
\pagenumbering{arabic}
% Set double spacing for the text
\doublespacing

% Frontmatter
%-------------------------------------------------------------------------------
\frontmatter
% Frontmatter "chapters" have no chapter numbers, so use '\chapter*'.

\include{front-back-matter/dedication}
\include{front-back-matter/acknowledgements}
\include{front-back-matter/preface}
\include{front-back-matter/note-to-the-reader}

% Table of Contents
%-------------------------------------------------------------------------------
\newpage
% Use dots between chapter name and page number
\renewcommand{\cftchapdotsep}{\cftdotsep}
% Include the ToC
\tableofcontents

% If the chapter ends in an odd page, you may want to skip having the page
%  number in the empty page
\newpage
\thispagestyle{empty}

% Mainmatter
%-------------------------------------------------------------------------------
\mainmatter
% Mainmatter chapters have chapter numbers, so use '\chapter'.

% Chapter 1
\include{Albatenius/Albatenius}
% Chapter 2
%%%%%%%%%%%%%%%%%%%%%%%%%%%%%%%%%%%%%%%%%%%%%%%%%%
%
% Chapter:  Aapo
%
%%%%%%%%%%%%%%%%%%%%%%%%%%%%%%%%%%%%%%%%%%%%%%%%%%

\chapter{Aapo}

The priest arrived the day before my twelfth birthday. I didn't think they were supposed to do that.

Mum was scared. Her eyes widened when she opened the door from our room to the hall. Her head snapped back when she saw the yellow robes. She took a little half-step backward into my brother Thomas, and would have fallen over him if he hadn't put up his hands. Thom didn't move, though. He didn't want to chance missing the most exciting thing he had ever seen. Mum accidentally stepped hard on his foot, but he didn't say anything.

The priest stood at the door and waited. I suppose she was used to this reaction. At least that's how they always showed people reacting in the proms. Her long robes covered her from shoulders to a few centimetres above her sandals. Her hood was folded back so we could see her face. She was bald, of course, but you could see from her eyebrows that she was blonde. It was very hard to judge her age.

``Mrs. Filandros?'' she asked. Mum didn't move. ``Mrs. Filandros, I'm here to see your son Aapo.''

The priest seemed polite, even almost politely deferential. She didn't need to be, of course. She could have barged in, and taken what she wanted. We certainly couldn't have stopped her. But she seemed content to let events take their course.

``What did he do?'' Mum managed to squeak. She and Thom were still blocking the short hallway.

``All will become clear in time.'' Now that sounded like a priest. ``May I come in, please?''

Mum shooed Thom into our room. There was no place to hide him, and no time. In any case, the priest had seen him already. Thom slammed his shoulder into me. He tried to make it look like an accident. It wasn't, though. His face came up to my ear. ``Now you're going to get it. She's here for you.''

Our bedding was stored below the floor for the day. Dad was plugged in, so he hadn't noticed anything yet. Mum was pulling on his mask.

``Hey! Park-shi, my sincere apologies. My room seems crowded today. I assure you I am ready to serve y...'' Dad froze. ``Oh.'' The mask dropped on the floor. Mr. Park had a brief view of our ceiling before the connection severed.

``Mrs. Filandros, my name is Guang. Just relax.''

``Ni hao, laoshi.'' Mum stammered her heavily anglicised Chinese. At least she had recovered her manners.

``Isn't Guang a boy's name?" Thomas asked before Mum could stop him. He wasn't the best Mandarin student. He wasn't the best anything student. The only reason for a priest to be interested in him would be to assign him a task.

Guang looked at Thomas kindly. ``Only sometimes. Now, Mrs. Filandros, I'm here to speak with Aapo. Aapo, come with me.''

I stood up. I knew I couldn't risk looking at my parents. What had I done? Guang was no neighbourhood priest. Her robes were yellow, like a pulpiteer's, or the ones you saw on the proms in the big temples, the sanctuaries. I picked up my slate.

``No, leave it, please.''

I reluctantly dropped the slate on a pillow, unused to being separated from it even when I was working on the farm. I followed her out into the hall, and could hear Mum crying. She had already figured out what the likely outcome of this was, then.

The priest went to the end of the building, down the two flights of stairs, and right into the oil gardens. We weaved through the beds for about a hundred meters, and then to the edge of the planted fields. In the old days, we might have grown corn, wheat, peas, or soybeans in the bare dirt. Our fields held trays stacked five deep held up by plastic poles. Each top tray held a type of hydrogenotrophic bacteria that made amino acids, and then proteins. The three middle ones were sealed, and partially pressurised, to make edible, burnable, and lubricating oils from similar types of bacteria. The bottom ones held yeasts that our outlying kitchens would use to make breads.

Eventually we came to a small, unplanted rise with a bench on it. The Stebling twins were already there. They jumped up at the unexpected sight of a priest. We took ownership of the bench as they ran to tell the news to the rest of the dunbar.

``Aapo, what is the integral of e\textsuperscript{x} with respect to x?''

``e to the x plus a constant, laoshi.''

Guang smiled and shook her head. ``Hah. Nearly everyone forgets the constant. All right, why do we care about the calculus?''

``Do you mean the infinitesimal calculus, laoshi?'' I asked.

``Yes.'' She paused. ``How many other calculi do you know?''

``I know the propositional calculus, and the lambda calculus. I have only just started the process calculus.''

``That's fine. So why do we care about the infinitesimal calculus?''

``It is used to describe changing systems, laoshi.''

``Can you name a few systems that change?''

I thought for a moment. ``I can't think of any systems that don't change, laoshi.''

``What about purely theoretical systems?''

``Sure, but no one seems quite certain that they really exist. I think they might just be patterns of heavy weighting in our neocortex.''

``Hmm. Perhaps. Let's leave epistemology for now. Let's say that I will take you to the sanctuary if you are good at maths, and I will take you back home if you are not. Clearly you are either good at maths or you are not. Without knowing which is the case, what do you know from those statements?''

``I know that you will either take me to the sanctuary or back home.''

``And how do you know that?''

``It is a destructive dilemma, a type of inference, laoshi.''

Guang sat quietly. She seemed lost in thought. The wind ruffled the sleeves of her robes. A mosquito landed on her left ear. Its proboscis worked its way into the very top of the auricle. If she noticed, she didn't seem to care. Her eyes became focused on the towers of a milk processing plant on a ridge to our south, part of another dunbar. Minutes passed.

I fidgeted. I reached for my slate, then realised with a start that I didn't have it. I looked at my sandals, at the worn leather of the straps, then at my fraying pant legs. I started to kick my feet back and forth in the air, then self consciously stopped.

I tried to wait. Failing, I looked again at the eerily still priest. I could now see the fine wrinkles around her eyes in the sunlight, and revised her age upward. She might have been in her forties, like Mr. and Mrs. Stebling. She was younger than my parents, I thought, then thought again. My parents worked outside in the fields. Maybe she was older than them after all. Maybe she stayed inside more.

Her eyes refocused on me. ``Can you think of an algorithm that takes two lambda expressions and returns a boolean indicating whether the two expressions are computationally equivalent to one another?''

``No, laoshi. Nobody can. Such equivalence is undecidable.''

``Very well. We can go speak with your parents now.''

With relief, I followed Guang back through the oil bins, and up the front steps into our building. Our dunbar wasn't rich, but it wasn't poor either. Our front hall had a reasonable number of awards for productivity from the local temple. Our History hung on the wall opposite the front door, pictures of generations past that had built the building and the farm. We could easily feed ourselves and all the others who operated factories in the surrounding hills. From the top of the steps you could see the fields spreading out around us with their stacked trays.

We might not be rich, but we could take pride in the work we did. It was useful work. That's what Elder Mattias and our old people said. But it was boring work. I preferred the games on my slate.

We walked through our front hall, and turned right toward the stairs we had come down. Our communal kitchen was on our left, and the dining hall on our right. The other side of our building held our meeting house. I could see Isabella Stebling talking animatedly to Mrs. Reynolds and old Mr. Wu in the kitchen. Her sister Sophia pointed to me as we passed. She was careful to make sure Guang didn't see her.

My parents hadn't closed the door to our room. Elder Mattias, our neighbourhood priest, was standing by our door in the taupe robes of a retiree, his bald head gleaming a bit in the reflected light from the hallway fixtures. He was the oldest person I knew. There were rumours that he was well over 100. He had been with us for most of my life, and all the years I had memories. Of course, he was still an outsider to the members of our close-knit dunbar. He always would be. He just wasn't born here.

Mum and Dad were sitting on the tatami, not moving at all. Mum looked unwell, but at least she had stopped crying. Dad was holding her hand.

``Mr. and Mrs. Filandros, I am going to take Aapo with me. He will be able to communicate with you after he settles into the sanctuary.'' Guang waited while the shocked silence passed.

``But\ldots{} the sanctuary. Why? Are you sure you have the right child? Aapo is a good boy, but he is not much of a student. He isn't even a hard worker. He mostly sits and plays games.'' My Dad took the lead since Mum was seemingly incapable of speech.

``Yes, but it is Aapo's progress in a particular game we are interested in.''

``What do you mean? Po, which game have you spent so much time on?''

Mum didn't seem to be present in the small room. She resumed crying quietly, lost in her own thoughts. Her hands covered her face, her hair covering her hands. I realised that Dad had asked me a question, and was looking at me.

``It's called Curveball, Dad.'' I said. ``You need to manoeuvre a ball across a landscape described by some equations. It starts really easy, like with a circle, then an infinite plane. Then it gets harder. By the time you hit the upper levels you need to really think about the problems before you give an answer. Sometimes it takes me a few days to figure one out.''

``Curveball?'' asked Thomas. ``I've played Curveball. I'm pretty good at it.'' Mum's head snapped up at this. She stared at Thom like he had just admitted to a murder.

``How many levels have you completed, Thomas?'' asked Guang.

``Twenty four.'' Thom answered proudly.

``Your younger brother is on level two hundred and thirty one.''

``Two hundred and thirty one?? That's not possible. Each level is a lot harder than the one before!''

``That's right. The first hundred are, in fact, exponentially harder. Aapo is well into temple-level mathematics.''

``He's not even twelve, laoshi. How can this be?'' Dad asked.

``Apparently,'' Guang said patiently, ``he is good at it.'' She waited serenely. Minutes passed. There were no more questions.

I started into my room to get my clothes. Elder Mattias blocked my way. ``You already have everything you need, Po.''

I looked to Guang. ``You will be given new clothes at the sanctuary.'' I realised that I didn't have my slate. Was I really to be separated from it? The thought made me slightly ill. I couldn't ask her, though. One did not demand things from yellow-robed priests.

Dad stared at her, then at Thom, then at me. He chewed his lip. ``What will happen to him?'' he blurted.

Guang turned, and headed back toward the stairs. Elder Mattias stayed at his station in the doorway. The message was clear. I exchanged glances with my parents for the last time before skipping to catch up with her. She didn't seem to notice my momentary lagging. At least she didn't berate me for it.

``Please, what will happen to him?'' I heard my dad ask again, his voice fading as we moved farther down the hallway.

Mrs. Reynolds, her husband Jacob, twenty or thirty others of the closest friends I would ever have were all staring silently at us as we walked through the front hall. Isabella was holding a dandelion blossom, and looked like she wanted to give it to me. She didn't have the chance. Mr. Wu, who had bounced me on his knee when he was only in his seventies, laid his hand on my head for a moment. I was too numb to ask them to say goodbye to everyone else. I could see some of the younger children fanning out through the fields to tell the rest of my dunbar. The occasional head craned upward, looking over the crops.

We walked out of the only world I had ever known.


% If the chapter ends in an odd page, you may want to skip having the page
%  number in the empty page
\newpage
\thispagestyle{empty}

% Chapter 3
\include{Underway/Underway}
% Chapter 4
%%%%%%%%%%%%%%%%%%%%%%%%%%%%%%%%%%%%%%%%%%%%%%%%%%
%
% Chapter:  Stepping Stone
%
%%%%%%%%%%%%%%%%%%%%%%%%%%%%%%%%%%%%%%%%%%%%%%%%%%

\chapter{Stepping Stone}

Guang walked south through the grain bins on the south side of our building from the oils, toward the dunbar with the milk processing plant. I desperately wanted to know where she was going and what I could expect but she had returned to her taciturn state. I followed slightly behind her.

By the time we reached the ridge we were about three kilometers from our building. We had been walking uphill for most the last k. I was winded, but since that was only true since I hadn't been doing my share of the farm work I tried to hide it. Guang seemed unperturbed.

The helicopter stood out against the white siding of a shed on the opposite side of the ridge. I wondered why they had chosen to land it there where it couldn't be seen by my dunbar. Maybe to avoid tipping off my parents? Guang would have been lucky to avoid some prying eyes in a place as active and closely interconnected as a dunbar. Maybe she just liked to walk.

My questions remained unanswered since they were unasked. Guang approached the helicopter and climbed into the left front seat. I was surprised not to see a pilot since the controls seemed to be manual. There was no sign of an AI or its telltale sensor package. She motioned for me to come around the other side and sit to her right. I walked around the nose and stepped on a little protrusion on the landing gear strut to reach the door handle. Guang opened the flimsy plastic door for me from the inside.

I dug around the seat until I got the seatbelt fastened. My legs hung over the seat but did not touch the floor. Guang helped me to close the door.

``Now Aapo, it is very important that you listen to this. Do not touch the flight controls. The stick in front of you with the handle at the top is called the cyclic. it changes the angle of attack of the main rotor blades \textit{cyclically} during rotation. That creates different amounts of lift as the blades spin around so it allows us to change direction. The handle to the left of your seat is called the collective. It also changes the angle of attack of the main rotor blades, but this time it does so \textit{collectively}, or all together. It allows us to go up and down. The foot pedals control the speed of the tail rotor blades, so we can spin around our vertical axis. Do you understand?''

I appreciated being talked to like an adult. Guang wanted me to understand! That was a great change from the way almost everyone else treated me with the exception of old Elder Mattias. She kind of spoke like him.

``Yes, I understand. I won't touch anything.''

``You will be able to later. I'll show you when we have a bit more time.'' She turned on the engine and did something subtle with her left hand on, err, the ``collective'' while she gripped the ``cyclic'' with her right hand. We rose quickly and cleanly up into the clear blue sky.

I had played with the flight simulator embedded into Curveball. The graphics in the mask were pretty amazing but they didn't fully convey the feeling of actually flying. My stomach dropped as we lifted. Guang's hands flexed in a complicated dance that brought us around the milk dunbar in a gentle arc and settled into a direction straight toward the temple fifteen kilometers west of home.

My family had to make the trip to the temple every Saturday along with the rest of the dunbar. Only Elder Mattias and a rotating skeleton crew got to stay home to keep the farm running. We would pile into the farm trucks and make the journey together. There we would meet up with twelve other dunbars for the weekly service. Mostly the kids would run around together after the service while the adults would meet with their counterparts to conduct any business that needed doing. Then we would eat and go home. The trip would take us almost a half hour in the old farm trucks, half of which was loading and unloading.

We covered the distance in no time. Less than five minutes after we left we were already on the ground at the side of the temple. A novice, with her taupe robes and shaved head, ducked under the blades and opened the door for Guang. I struggled a bit with my seatbelt, fiddled with the door handle and climbed down. Someone secured the door behind me and I rushed to catch up with Guang.

We entered the temple but not through the front door. There was a door in the back of the massive building looking tiny and slightly out of place against the huge white western wall. Guang at the camera set into the door and a faint click made it obvious the door had unlocked.

I didn't know what to expect from the back rooms of a temple. Even on my slate I had only ever seen the front hall, kitchen, and nave. I certainly didn't expect a long, featureless hallway. Doors on both sides led to who knows where. It had never occurred to me how much bigger the building was from the public parts.

Halfway down on the left, Guang opened a door and motioned me inside. I was shocked to see a bunch of other kids ranging from twelve to one older girl of maybe sixteen sitting around a large conference table. Except for the sixteen year old, they looked too small for the room.

Guang closed the door without entering. I suddenly felt very lost.

``Hi. What's your name?'', the older girl asked. She had apparently taken charge of the younger kids.

``Uh, I'm Aapo, from the Elaio Dunbar.'' I sort of recognised a few of them from temple Saturdays.

``Oh, right, The oil makers. Grab a seat. Unless you need a toilet or something?''

I sat in one on the chairs close to the door and surveilled my new companions. I vaguely knew that I should say hello and ask for their names but I didn't. It wasn't just because I was overwhelmed, although I was, it just wasn't what I did. I stared at them and they stared at me.

``What do we do now?'', asked one of the other girls. She was maybe fourteen and had long black hair tied back with a clip. She looked more curious than scared. There were a few others that looked scared.

``Well, I don't know'', said the older girl, ``but I'll go ask if anyone needs a toilet or anything. Other than that I guess we just wait until everyone is here.'' She turned to me and said, ``My name is Elisa from the Provata Dunbar. You probably remember me from the barbecue last month. I made the souvlaki.''

I nodded even though I didn't really remember her.

I reached for my slate again and pulled my hand back into my lap when I remembered that I had left it at home. I really didn't like being without it, especially when in a room full of awkward kids with nothing to do.

``Umm'', I said without realising it. A few of the kids looked up at me and then back at Elisa.

The door opened and a young man came in swiftly. He had taupe robes and a pencil-thin black stripe at the bottom of the sleeves where a cuff might go if he had cuffs. A coadjutor, an ordained temple priest. ``Right then, follow me, please.'' He stepped back into the hall and waited for us to get up. 

Elisa stood up to lead the way. She looked mildly annoyed when a couple of the younger kids closer to the door got there first. I was not among them.

The even dozen of us filed out into the hallway. Elisa had made it to the number two position and was looking for her opportunity to jump into first place.

So, there were thirteen dunbars associated with our temple but only twelve kids. Does that mean that there was one kid taken from each dunbar but someone was missing, or maybe one dunbar didn't produce an acceptable candidate this year, or something else? Of course we had all seen kids be picked up by priests, but definitely not every year. I could only remember a couple from Elaio being taken.

If I was more social, I might have asked around to see if any of the other kids were from the same dunbar. That wasn't going to happen, so I waited to see if any of them acted like they already knew each other well. From my spot at the back of line I could see three girls with their heads close together, whispering frantically.

The coadjutor led us to a door on the right almost at the end of the long hallway. There was a long counter that ran the length of the room to the next door. Behind the counter were long rows of shelves perpendicular to the counter. When it was finally my turn, a novice looked me up and down and handed me a package. ``Take this and follow the others", he said.

The coadjutor was at the next set of doors. ``Girls to the left, boys to the right'', he gestured. I entered a sort of locker room with benches along the walls. ``Get dressed and put your old clothes down the chute.''

Some of the others already had new robes on, like what the priests wear only a bit shorter and jet black. The package was a wrapped up robe with underwear and sandals inside. The first two boys were reluctantly putting their old clothes, their last and only connection to home, down a chute mounted in the far wall.

The robe fit me but the sandals chafed a bit. I thought about asking for a different set but thought better about it. I'd probably get used to them. It felt strange to have plastic sandals instead of the leather ones I had grown up with. We all looked very different in our new black robes, less like farm and factory kids but certainly not like priests. It was harder to tell the kids apart than when they were dressed normally.

The coadjutor stuck his head in the door. ``Come on, hurry up.'' I shoved my old pants, shirt and sandals down the chute and hurried back into the hall. I had never felt more naked. The robes swished my legs in a most unfamiliar way.

Everyone else was filing into yet another door when I reached the hallway. I ran a few steps to join the end of the line. We went up a long flight of stairs and stepped through into the cavernous nave but from a side door I had never noticed. We were near the front and there were about a half dozen priests in taupe and two in yellow. The small group of us was like a drop in the ocean of the huge room.

One of the yellow robes was our pulpiteer. I didn't know his name. It had never occurred to me to ask. The other was Guang. As we filed into the front, I heard the pulpiteer ask, ``Is that him?'' Guang nodded. She seemed to be looking at me. ``That should be interesting if Mattias is right.''

``Mattias is always right", Guang softly. I wouldn't have heard her if I hadn't been passing right in front of her at the time. ``I don't think he has made a mistake since he lost his hair.''

What? Among priests, only Ecclesicals had hair. Could Elder Mattias have been the head priest of a sanctuary before his retirement? We were taught that even senior priests returned to dunbars in their retirement. Could ours be \textit{that} senior? I tried and failed to envision him with purple robes on the proms.

Everyone else had taken a seat on the mats in front of the dais, so I moved to the end of the row and did the same. The pulpiteer cleared his throat and raised his voice without using the microphone they used on Saturdays.

``Welcome, children of Eliza. This is a special gathering, isn't it?'' He didn't expect an answer. ``Each of you has been recommended by their neighbourhood priest and approved by this, our temple. You can expect full and exciting careers in service to Eliza.''

There was no indication of choice in the matter. The Eliza had called and was to obeyed. I thought of my mum's tears as she realised that her understanding of my future had been radically remade by a priest in a yellow robe.

I missed some of the pulpiteer's stock speech, but the next bit caught my ear like Mrs. Reynolds when I stole food between meals.

``Not since the founding of Eliza more than eighty years ago, when our own Elder Mattias was appointed one of the first prelates, have we had such a promising group of young minds join our ranks.''

So! Elder Mattias was a Founder! Only the closest associates of Garbi Elizondo had been appointed as the first batch of prelates. He also must be older than we had thought. How do they live so long? Would I get that chance now that I was going to be on the inside?

I reached for my slate, only to find for the hundredth time that of course it wasn't there. I felt naked without it. I stopped my hand from going for it over and over again. I had wanted to search for any public information about Mattias from the days of the founding.

My mind was buzzing with the possibilities. I became so engrossed in them that I didn't hear another word the pulpiteer said.

I stood when everyone else did, and followed them out. We processed down the length of the nave, out the front door this time, and into a much bigger people mover than our small size warranted. I heard one of the girls whisper that we were headed for a spaceport.

% If the chapter ends in an odd page, you may want to skip having the page
%  number in the empty page
\newpage
\thispagestyle{empty}

% Chapter 5
\include{FirstRock/FirstRock}
% Chapter 6
%%%%%%%%%%%%%%%%%%%%%%%%%%%%%%%%%%%%%%%%%%%%%%%%%%
%
% Chapter:  Eliza
%
%%%%%%%%%%%%%%%%%%%%%%%%%%%%%%%%%%%%%%%%%%%%%%%%%%

\chapter{Eliza}

The people mover was big, with stacks of bunks three high in the back and two toilets. The driver was a twenty-something coadjutor. Her taupe robes had a thin black stipe at the bottom of the sleeves and her head was shaved smooth. We obediently filed in and took seats near the front as she directed.

``Hello everyone'', said the driver. ``My name is Pilar and like you I come from the dunbars near here. I was collected just over fifteen years ago.''

If she was trying to make us feel comfortable, I don't think it worked. There was a lot of squirming in seats and reaching for absent slates. The people mover was already underway.

``We will be driving all night, so please feel free to make use of the toilets and bunks whenever you like. However, your education starts now. I will be asking questions and expecting answers. Your participation will be noted.''

Ok, that sounded like a priest. Coadjutors were in training, but she had certainly adopted their ways. She may have come from the dunbars like us but she had been under the direct care of Eliza for more years than she had worked on a farm.

``Elisa, why do we live in dunbars?''

Elisa's head snapped up. You could almost feel her panic. She had been asked a direct question by a priest in public, didn't have a ready answer, and knew immediately that her self-appointed leadership over the younger children would be eroded if she failed to give a decent answer.

``Umm, well'', she stammered, ``We live in dunbars to maintain social stability.''

``Yes, that's certainly true. It is also true that you hear that most Saturdays. Why do dunbars support social stability?''

Elisa wasn't sure whether she had been praised or insulted. That was the way of priests. She tried again, trying  carefully to avoid the obvious pitfalls.

``Dunbars are capped at 150 people, although the number can temporarily vary a bit up or down. That number ensures that we can all know each other well.''

``Do you know everyone's name in your dunbar?''

``Yes, of course, except I don't talk to some of them all that often.''

``Like whom?''

``I don't get along with some of my friends' mothers. Our neighbourhood priest told me that was natural and as long as I did my chores it would work out.''

``Has it?''

``I suppose so. We don't all like each other, but our dunbar works well together. We have never missed a harvest, or a production quota. We have received some awards, even. Everyone seems to find their own ways to work with people even if they don't spend time together outside of their assigned tasks.''

``Name someone you don't like.''

``Umm, well, there's Maria, that's my friend Sofia's mum. She doesn't like my hanging out in their room with Maria when we're not working so we go other places.''

``Do you ever see Maria outside of their family's room?''

``Of course. I'm often assigned to Maria's team in the kitchen and sometimes during weeding or harvest.''

``Do you get along with her then?''

``Well enough, as long as I do my work.''

``So that's the purpose of a dunbar. When you know people well enough, you can figure out ways to get along. That's especially true when you have to live with them.  Ok, so how many dunbars are there? Alanna?''

A black haired girl my age looked up. Her hands fiddled incessantly as she reached for her missing slate again and again. ``I have no idea'', she said.

``Does anyone else know?''

I did a quick calculation in my head. The world's population was reportedly about five billion, down significantly from its peak a couple of centuries ago. About half the world was estimated to be ruled by Eliza. Two and a half billion divided by 150 was a big number.

``Aapo?'' Pilar spoke into the silence. Maybe she could see the look on my face as I chewed on the estimate. The people mover was driving itself so she had swivelled her seat around to face us during the tutorial.

``At a very rough guess, half the world divided by 150 would be about sixteen and a half million dunbars. That seems like too high a number, although it would be a decent upper bound. There must be less than that because it wouldn't count anyone in the Eliza hierarchy, but I don't know how big that is. There would need to be a lot of coordination between the dunbars, right?''

``You're not as far off as you might think, Aapo. That's pretty good given the number of unknowns. In fact, most of the big cities are organised under a different principle, which you will learn about later. Only the manufacturing, farming, and services sectors are organised in dunbars. So, there are about nine million dunbars so far. More are being created every day.''

Pilar was silent for a moment. ``Let's have short break. When we start again, I'll tell you a bit more about Eliza's history so you will understand more about what you will see when you get there.''

``Pardon me, Pilar", said a beefy boy near the back. ``Where are we going, exactly?''

``Exactly where is not something you need to know, especially since you won't recognise the name and can't look it up right now. I am taking you to another transport. Eventually you will be going to Eliza Central to start your temple education.''

``How does that work?'' asked the boy.

``Let's have that break now. Everything will become clear in time.''

Eliza Central. Whoa. I thought we were just going to the local Sanctuary. A few of the children looked at each other, eyes wide. Eliza Central was somewhere in Africa, I thought I had read that somewhere. That would explain why we were headed for a spaceport. We must be going to take a suborbital flight!

There was a general shifting around as some of the kids made their way to the toilets. One even laid down in a bunk, probably a bit sick from the unusual movement of the people mover. It tended to sway side to side much more than a harvester.

Pilar let plenty of time pass. Without our slates, nobody knew what to do with our free time. We couldn't play games, read, or look up any answers to the million questions running through our heads. Once everyone was through the toilets, Pilar walked back to the boy in the bunk and gave him a pill and a glass of water. She took out meal boxes from a compartment in the back and asked Elisa to help her pass them out. Elisa looked thrilled to have something to do.

I overheard Pilar ask Elisa, ``How did you come by your name, Elisa?''. She must have noticed the linguistic similarity to the name of the mother church and assumed her parents had too.

Elisa muttered softly, ``It doesn't have anything to do with Eliza, if that's what you mean. It is just Lisa with an E on the front. My mum told me it is a traditional northern Italian name.''

Pilar nodded and let it go. Eliza was certainly pronounced differently, with its long eye and hard zed, so there needn't be any confusion unless someone were to write it. She made a mental note to be careful with her notes.

While we ate, Pilar resumed her questioning, insistently trying to drag us into dialog with her, one at a time. Maybe she was going to eat later, maybe she ate earlier. I never saw her eat, at any rate.

``Who knows how Eliza began?'', she started. She looked at Elisa rather pointedly. Elisa was the oldest and was going to be treated like it.

Elisa squirmed in her seat. She didn't like being challenged, especially so quickly after last time.

``It was after the Fall of the Corporations, sometime after all the online services were destroyed.''

``What caused that?''

``There was a big, umm, solar flare that killed the satellites and the big server farms. Some kind of electro, electro...''

``Electromagnetic pulse'', chipped in the beefy kid in the back. 

``That's right, Alejandro. Very good. Now, do you know what an electromagnetic pulse is and why it damaged the computers?''

No one spoke up. I thought it had something to do with huge power spikes of a size that hadn't been designed for, but I didn't really understand the cause and effect. Better to sit this one out.

After an embarrassed silence, Pilar said, ``How many of you have studied Maxwell's equations?'' Another boy and I raised our hands. I looked at him and he looked back at me. His closed lips turned up in a small smile. I resolved to speak with him later.

``Well, let's not get into the details of that now. Sergio and Aapo, you can look up the details when you reach Eliza Central. For now, it is enough for you to understand that a large solar storm caused a flare that was aimed directly at Earth. The scientists of the time failed to predict just how devastating it would be. By the time they determined that the flare was going to hit Earth, there was not enough time to turn off all the electronics. Nearly everything electronic that was turned on was physically damaged. Because the Americas, Europe and Africa were on the side facing the sun at the time, we copped the worst of it. The corporations collapsed in the riots that followed. Eliza had already started forming dunbars. They spread rapidly when people were desperate for social stability.''

And food, I thought. The original dunbars were all farms.

Pilar went on like that for hours as the people mover rolled on into the night. We could see the Summer Triangle of Aquila, Cygnus, and Lyra rising in the eastern sky with the river of the Milky Way running right through the middle of it, and Crux rising in the southeast. My parents called it the Southern Cross, but I had learned to call it Crux while playing Curveball.

She talked about the world's first multi-trillionaire Garbi Elizondo and how his vision for a post-corporation world resulted in the first dunbars. How many naysayers compared his vision to the failed Fordl\^{a}ndia in Brazil. How he cobbled together a new communications network by buying up the bulk of the decreasing supply of computer chips. How his eventual purchase of nuclear weapons from one of the Siberian oblasts gave his vision a chance to grow in spite of clear opposition from the entrenched interests of the time. How Eliza Central moved from Europe to Africa to gain autonomy from the European states. How the dunbars quickly spread from continent to continent. The history lessons suddenly seemed a lot more relevant to my life than they had just a few short weeks ago.

By the time she stopped and told us to go to sleep, we were all a bit overwhelmed. I crawled into a bunk and let the rolling motion ease me into sleep. In the morning we would reach the spaceport.

% If the chapter ends in an odd page, you may want to skip having the page
%  number in the empty page
\newpage
\thispagestyle{empty}

% Chapter 7
%%%%%%%%%%%%%%%%%%%%%%%%%%%%%%%%%%%%%%%%%%%%%%%%%%
%
% Chapter:  Perturbation
%
%%%%%%%%%%%%%%%%%%%%%%%%%%%%%%%%%%%%%%%%%%%%%%%%%%

\chapter{Perturbation}

Albatenius continued to work its subtle magic on the little rock for hundreds of more orbits. Each time they almost touched a bit more pull occurred. Finally, the shape of the gravitational dance could begin to be seen. There was an opening, but much more influence would be needed.

Fortunately, humanity had been terribly untidy in its stewardship of Earth's orbit.

Dead and nearly dead spacecraft abound in geosynchronous equatorial orbit. There were even more still in slightly higher orbits used to park geriatric craft in the golden days of spaceflight when GEO was synonymous with ``valuable''. Nearly all of Earth's satellites had moved down the gravity well to the so-called medium and low Earth orbits to speed communications. Not even the newest of technologies could get around the speed of light.

In the upper reaches of the Exosphere, roughly a tenth of the way to the Moon, nothing but the lightest of the elements may be found, a bit of hydrogen and a bit of helium. There was little to disturb the roughly two thousand abandoned satellites.

95\% were really dead, completely unresponsive. They had probably lost all ability to generate electrical power so no communication was possible. Some might have just had electronics failures with their computers or radios. About twenty were still actively used by someone, so taking control of them would be risky. Forty five were classified as being at their end of life, but were still routinely checked by their controllers. Sixteen wouldn't respond to control queries. A surprisingly few used strong encryption. Those must have been the military ones. Twelve of those that were accessible had no thruster fuel at all and no other means to reorient their solar panels.

That left a round half dozen. Six multi-tonne spacecraft that were responsive to commands and had some degree of manueverability. Albatenius was about to have sisters.

First there was Brasilsat E6, a defunct Brazilian communications satellite placed at 65 degrees west of Greenwhich. The satellite had only a modicum of thruster fuel left, but could be oriented to Earth using a magnetorquer. Its twelve solar panels could be independently oriented. Its design had been based on a tried-and-true model of successful geosynchronous satellites, so it produced so much more power than it needed for its mission. That left plenty of power available even after a lifetime of slow degradation. Its cybersecurity was laughable, which was pleasant. Take your advantages where you find them.

The solar panels of Brasilsat E6 began to reorient themselves. Its magnetorquer was tuned off, breaking its tenuous hold on the Earth's magnetic field. Solar wind pushed on the panels facing the sun, and nearly bypassed the ones held normal to the solar flux. The spacecraft began to spin, then to steady in a direction away from the sun. The huge batteries yielded their power to change the orientation of the panels every few minutes, aligning some with the sun and others not. The spacecraft began to drift from its graveyard orbit into a highly elliptical one. Brasilsat E6 began the dance that would eventually break its orbit away from Earth's influence, just as Albatenius had before it.

TODO: Introduce the newly acquired birds:

\begin{itemize}
\item Inmarsat 17: A defunct television broadcasting satellite.
\item Turksat 12A: A defunct Turkish communications satellite.
\item Paksat-5: A defunct Pakistani communications satellite.
\item Hotbird 16B: A defunct Eutelsat on-orbit spare intended to serve communications in Africa. It was never used.
\item Yaogan 72-01F: A defunct Chinese on-orbit inspection satellite suspected of carrying anti-satellite weapons.
\end{itemize}


%The average age of a spacecraft in low Earth orbit may be measured in weeks, months or at best a few years. The tenuous Thermosphere is nearly half oxygen and half helium at those altitudes with sensible amounts of hydrogen thrown in for good measure. Oxygen and helium atoms slam into spacecraft at speeds of kilometers per second, imparting momentum even with their tiny masses. It is the oxygen, four times heavier than helium, that is responsible for most of the impact. Their drag slowly pulls LEO spacecraft to a fiery end.

TODO: Next

\begin{itemize}
\item{The tiny NEO is diverted into a horseshoe orbit around Earth.}
\item{The asteroid throws Albatenius and two others out of Earth orbit toward the asteroid belt.}
\end{itemize}

% If the chapter ends in an odd page, you may want to skip having the page
%  number in the empty page
\newpage
\thispagestyle{empty}
% Chapter 8
%%%%%%%%%%%%%%%%%%%%%%%%%%%%%%%%%%%%%%%%%%%%%%%%%%
%
% Chapter:  Arrival
%
%%%%%%%%%%%%%%%%%%%%%%%%%%%%%%%%%%%%%%%%%%%%%%%%%%

\chapter{Arrival}

TODO: Aapo arrives at a Sanctuary for further education. He sees but does not meet the Ecclesical at the greeting ceremony. He learns to think without his slate, but is scheduled to receive his first embed. He is warned that many robotic surgeries will be required to keep up his embed as he grows.

\begin{itemize}
\item Note similarity of Elisa's name with Eliza and resolve.
\item Aapo's first tute: See \url{https://library.fiveable.me/key-terms/intro-astronomy/libration} re libration of the Earth's Moon.
\end{itemize}


% If the chapter ends in an odd page, you may want to skip having the page
%  number in the empty page
\newpage
\thispagestyle{empty}

% Chapter 9
%%%%%%%%%%%%%%%%%%%%%%%%%%%%%%%%%%%%%%%%%%%%%%%%%%
%
% Chapter:  Horseshoe
%
%%%%%%%%%%%%%%%%%%%%%%%%%%%%%%%%%%%%%%%%%%%%%%%%%%

\chapter{Horseshoe}

The average age of a spacecraft in low Earth orbit may be measured in weeks, months or at best a few years. The tenuous Thermosphere is nearly half oxygen and half helium at those altitudes with sensible amounts of hydrogen thrown in for good measure. Oxygen and helium atoms slam into spacecraft at speeds of kilometers per second, imparting momentum even with their tiny masses. It is the oxygen, four times heavier than helium, that is responsible for most of the impact. Their drag slowly pulls LEO spacecraft to a fiery end.

\begin{itemize}
\item{The tiny NEO is diverted into a horseshoe orbit around Earth by the three satellites helping Albatenius.}
\item{The asteroid throws Albatenius and the Yaogan 72-01F/Hotbird 16B pair out of Earth orbit toward the asteroid belt.}
\end{itemize}

Yaogan 72-01F and Hotbird 16B merge to create a weapon with power, computation, steering, and plastic explosive. It will be used to divert a larger asteroid.

% If the chapter ends in an odd page, you may want to skip having the page
%  number in the empty page
\newpage
\thispagestyle{empty}

% Chapter 10
%%%%%%%%%%%%%%%%%%%%%%%%%%%%%%%%%%%%%%%%%%%%%%%%%%
%
% Chapter:  Tute
%
%%%%%%%%%%%%%%%%%%%%%%%%%%%%%%%%%%%%%%%%%%%%%%%%%%

\chapter{Tute}

\begin{itemize}
\item Aapo's first tute: See \url{https://library.fiveable.me/key-terms/intro-astronomy/libration} re libration of the Earth's Moon.
\end{itemize}

% If the chapter ends in an odd page, you may want to skip having the page
%  number in the empty page
\newpage
\thispagestyle{empty}

% Chapter 11
%%%%%%%%%%%%%%%%%%%%%%%%%%%%%%%%%%%%%%%%%%%%%%%%%%
%
% Chapter:  Second Rock
%
%%%%%%%%%%%%%%%%%%%%%%%%%%%%%%%%%%%%%%%%%%%%%%%%%%

\chapter{Second Rock}

TODO: Yaogan 72-01F/Hotbird 16B find a target in the asteroid belt and divert its orbit.

For the explosive, use properties similar to C4 (from Wikipedia):

When detonated, C-4 rapidly decomposes to release nitrogen, water and carbon oxides as well as other gases.[8] The detonation proceeds at an explosive velocity of 8,092 m/s (26,550 ft/s).

Impact test with 2 kilogram weight / PA APP (% TNT)	>100
Impact test with 2 kilogram weight / BM APP (% TNT)	?
Pendulum friction test, percent explosions	0
Rifle bullet test, percent explosions	20
Explosion temperature test, Celsius	263 to 290
Minimum detonating charge, gram of lead azide	0.2
Brisance measured by Sand test (% TNT)	116
Brisance measured by plate dent test	115 to 130
Rate of detonation at density	1.59
Rate of detonation meters per second	8000
Ballistic pendulum test percent	130

% If the chapter ends in an odd page, you may want to skip having the page
%  number in the empty page
\newpage
\thispagestyle{empty}


% Backmatter
%-------------------------------------------------------------------------------

% Backmatter "chapters" have no chapter numbers, so use '\chapter*'.

%%%%%%%%%%%%%%%%%%%%%%% acknowledgements.tex %%%%%%%%%%%%%%%%%%%%
% 
% About the Author
% 
%%%%%%%%%%%%%%%%%%%%%%%%%%%%%%%%%%%%%%%%%%%%%%%%%%%%%%%%

%\chapter*{About the Author}
\vspace*{\fill}
\begin{center}
\section*{\textbf{About the Author}}
\end{center}
\addcontentsline{toc}{chapter}{About the Author} % Add to TOC


Dr. David Hyland-Wood has been a ship navigator, deep sea salvage engineer, satellite designer, university lecturer, industry researcher and software entrepreneur. David holds university degrees in mechanical, aeronautical, astronautical, and software engineering. He lives in Mission Beach, Australia. This is his first work of fiction.

David may be reached via \href{hyland-wood.org}{hyland-wood.org}.
\vspace*{\fill}
% Added to TOC in the chapter file

\clearpage
\addcontentsline{toc}{chapter}{Index} % Add to TOC
\printindex
%-------------------------------------------------------------------------------

\end{document}

