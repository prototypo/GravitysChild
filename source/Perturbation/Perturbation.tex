%%%%%%%%%%%%%%%%%%%%%%%%%%%%%%%%%%%%%%%%%%%%%%%%%%
%
% Chapter:  Perturbation
%
%%%%%%%%%%%%%%%%%%%%%%%%%%%%%%%%%%%%%%%%%%%%%%%%%%

\chapter{Perturbation}

Albatenius continued to work its subtle magic on the little rock for hundreds of more orbits. Each time they almost touched a bit more pull occurred. Finally, the shape of the gravitational dance could begin to be seen. There was an opening, but much more influence would be needed.

Fortunately, humanity had been terribly untidy in its stewardship of Earth's orbit.

Dead and nearly dead spacecraft abound in geosynchronous equatorial orbit. There were even more still in slightly higher orbits used to park geriatric craft in the golden days of spaceflight when GEO was synonymous with ``valuable''. Nearly all of Earth's satellites had moved down the gravity well to the so-called medium and low Earth orbits to speed communications. Not even the newest of technologies could get around the speed of light.

In the upper reaches of the Exosphere, roughly a tenth of the way to the Moon, nothing but the lightest of the elements may be found, a bit of hydrogen and a bit of helium. There was little to disturb the roughly two thousand abandoned satellites.

95\% were really dead, completely unresponsive. They had probably lost all ability to generate electrical power so no communication was possible. Some might have just had electronics failures with their computers or radios. About twenty were still actively used by someone, so taking control of them would be risky. Forty five were classified as being at their end of life, but were still routinely checked by their controllers. Sixteen wouldn't respond to control queries. A surprisingly few used strong encryption. Those must have been the military ones. Twelve of those that were accessible had no thruster fuel at all and no other means to reorient their solar panels.

That left a round half dozen. Six multi-tonne spacecraft that were responsive to commands and had some degree of manueverability. Albatenius was about to have sisters.

First there was Brasilsat E6, a defunct Brazilian communications satellite placed at 65 degrees west of Greenwhich. The satellite had only a modicum of thruster fuel left, but could be oriented to Earth using a magnetorquer. Its twelve solar panels could be independently oriented. Its design had been based on a tried-and-true model of successful geosynchronous satellites, so it produced so much more power than it had needed for its mission. That left plenty of power available even after a lifetime of slow degradation. Its cybersecurity was laughable, which was pleasant. Take your advantages where you find them.

The solar panels of Brasilsat E6 began to reorient themselves. Its magnetorquer was tuned off, breaking its tenuous hold on the Earth's magnetic field. Solar wind pushed on the panels facing the sun, and nearly bypassed the ones held normal to the solar flux. The spacecraft began to spin, then to steady in a direction away from the sun. The huge batteries yielded their power to change the orientation of the panels every few minutes, aligning some with the sun and others not. The spacecraft began to drift from its graveyard orbit into a highly elliptical one. Brasilsat E6 began the dance that would eventually break its orbit away from Earth's influence, just as Albatenius had before it.

Three defunct television and communication relay satellites were next: Inmarsat 17, Turksat 12A, and Paksat-5. All three had a bear minimum of hydrazine frozen to the inside of their thruster tanks, had some reasonable battery life left, and had functional, if badly degraded solar panels. They were near similar enough in design to be close cousins, having all been manufactured by the same European conglomerate. They would all stay in high Earth orbit, but gentle nudges began moving them into orbits that were ever more highly elliptical. Eventually they would also interact with the tiny asteroid that Albatenius was playing with. 

Fifth was Hotbird 16B, one of several Eutelsat on-orbit spares intended to eventually serve communications in Africa. Amazingly, it had never been used. Its tanks were nearly full, missing only a bit of boil off. Its solar panels had degraded as much as the rest of the fleet, but its batteries were in excellent condition. Hotbird 16B was a literal gold mine of resources in the graveyard of dead and nearly dead space junk. Its owners would miss it but almost never checked on its status. A few tweaks to the databases of the Eutelsat operations centre should be enough to avoid those checks becoming necessary. 

Lastly, there was Yaogan 72-01F, a defunct Chinese on-orbit inspection satellite that had been suspected of carrying anti-satellite weapons. It had no battery life to speak of, its thruster fuel had long since been depleted, but its solar panels did still produce a current. Critically, it had two items onboard that no other satellite in the small fleet had; there was a sort of long, narrow probe that would fit into the transfer rocket nozzle of a geosat and open to physically grab onto it, and a wonderful surprise. Hidden from outside view, buried deep in the satellite guts between the thruster tanks, the depleted transfer tank, the batteries and the radios, accessible only from an encrypted folder in its operating system, lay a block of plastic explosive 2 kilograms in mass.

The explosive was similar to C4, that ever-present staple of American firepower since the mid-twentieth century. The differences were minor. There was a small amount of plasticizer, just enough to fit the explosive into an aluminium container in the recesses of the satellite. There was no need for an odorizing agent. So-called taggant chemicals are commonly added to explosives on Earth so they may be detected by specialist equipment. In space, such agents would only offgas into the vacuum, risking all sorts of contamination of the satellite's components. Yaogan 72-01F had been ready to destroy any satellite it could rendezvous with. Now it would be put to a very different purpose indeed.

% If the chapter ends in an odd page, you may want to skip having the page
%  number in the empty page
\newpage
\thispagestyle{empty}