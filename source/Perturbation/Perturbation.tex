%%%%%%%%%%%%%%%%%%%%%%%%%%%%%%%%%%%%%%%%%%%%%%%%%%
%
% Chapter:  Perturbation
%
%%%%%%%%%%%%%%%%%%%%%%%%%%%%%%%%%%%%%%%%%%%%%%%%%%

\chapter{Perturbation}

Albatenius continued to work its subtle magic on the little rock for hundreds of more orbits. Each time they almost touched a bit more pull occurred. Finally, the shape of the gravitational dance could begin to be seen. There was an opening, but much more influence would be needed.

Fortunately, humanity had been terribly untidy in its stewardship of Earth's orbit.

Dead and nearly dead spacecraft abound in geosynchronous equatorial orbit. There were even more still in slightly higher orbits used to park geriatric craft in the golden days of spaceflight when GEO was synonymous with ``valuable''. Nearly all of Earth's satellites had moved down the gravity well to the so-called medium and low Earth orbits to speed communications. Not even the newest of technologies could get around the speed of light.

In the upper reaches of the Exosphere, roughly a tenth of the way to the Moon, nothing but the lightest of the elements may be found, a bit of hydrogen and a bit of helium. There was little to disturb the roughly two thousand abandoned satellites.

95\% were really dead, completely unresponsive. They had probably lost all ability to generate electrical power so no communication was possible. Some might have just had electronics failures with their computers or radios. About twenty were still actively used by someone, so taking control of them would be risky. Forty five were classified as being at their end of life, but were still routinely checked by their controllers. Sixteen wouldn't respond to control queries. A surprisingly few used strong encryption. Those must have been the military ones. Twelve of those that were accessible had no thruster fuel at all and no other means to reorient their solar panels.

That left a round half dozen. Six multi-tonne spacecraft that were responsive to commands and had some degree of manueverability. Albatenius was about to have sisters.

First there was Brasilsat E6, a defunct Brazilian communications satellite placed at 65 degrees west of Greenwhich. The satellite had only a modicum of thruster fuel left, but could be oriented to Earth using a magnetorquer. Its twelve solar panels could be independently oriented. Its design had been based on a tried-and-true model of successful geosynchronous satellites, so it produced so much more power than it needed for its mission. That left plenty of power available even after a lifetime of slow degradation. Its cybersecurity was laughable, which was pleasant. Take your advantages where you find them.

The solar panels of Brasilsat E6 began to reorient themselves. Its magnetorquer was tuned off, breaking its tenuous hold on the Earth's magnetic field. Solar wind pushed on the panels facing the sun, and nearly bypassed the ones held normal to the solar flux. The spacecraft began to spin, then to steady in a direction away from the sun. The huge batteries yielded their power to change the orientation of the panels every few minutes, aligning some with the sun and others not. The spacecraft began to drift from its graveyard orbit into a highly elliptical one. Brasilsat E6 began the dance that would eventually break its orbit away from Earth's influence, just as Albatenius had before it.

TODO: Introduce the newly acquired birds:

\begin{itemize}
\item Inmarsat 17: A defunct television broadcasting satellite.
\item Turksat 12A: A defunct Turkish communications satellite.
\item Paksat-5: A defunct Pakistani communications satellite.
\item Hotbird 16B: A defunct Eutelsat on-orbit spare intended to serve communications in Africa. It was never used.
\item Yaogan 72-01F: A defunct Chinese on-orbit inspection satellite suspected of carrying anti-satellite weapons.
\end{itemize}


%The average age of a spacecraft in low Earth orbit may be measured in weeks, months or at best a few years. The tenuous Thermosphere is nearly half oxygen and half helium at those altitudes with sensible amounts of hydrogen thrown in for good measure. Oxygen and helium atoms slam into spacecraft at speeds of kilometers per second, imparting momentum even with their tiny masses. It is the oxygen, four times heavier than helium, that is responsible for most of the impact. Their drag slowly pulls LEO spacecraft to a fiery end.

TODO: Next

\begin{itemize}
\item{The tiny NEO is diverted into a horseshoe orbit around Earth.}
\item{The asteroid throws Albatenius and two others out of Earth orbit toward the asteroid belt.}
\end{itemize}

% If the chapter ends in an odd page, you may want to skip having the page
%  number in the empty page
\newpage
\thispagestyle{empty}